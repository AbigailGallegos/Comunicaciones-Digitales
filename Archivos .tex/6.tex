\begin{center}
	\LARGE\bfseries CAD
\end{center}

El objetivo principal de los convertidores A/D dentro de un sistema de adquisición de datos es convertir las señales analógicas condicionadas en un flujo de datos digitales para que el sistema de adquisición de datos pueda procesarlos para su visualización, almacenamiento y análisis.

Algunos tipos de CAD comunes en la actualidad:

\begin{itemize}
	\item ADC de aproximación sucesiva (SAR)
	\item	ADC Delta-sigma  $\Delta\Sigma$
	\item ADC de doble pendiente
	\item ADC en línea
	\item ADC Flash
\end{itemize}

Características importantes en un CAD : \\

La velocidad a la que se convierten las señales del dominio analógico a un flujo de datos digitales se denomina \textbf{frecuencia de muestreo} o \textbf{sample rate}, esta depende de la aplicación. La frecuencia de muestreo suele denominarse T (tiempo) o eje X de medición.\\

\textbf{Resolución de bits}, o un número de bits implica el eje de amplitud (Y). La apariencia de las formas de onda es, en consecuencia, más exacta y tiene mucha más precisión, cuanta más resolución tenga. Esto se aplica también al eje temporal.

 \begin{figure}[!htbp]
	\subfloat[Sample rate]{
		\includegraphics[height=6cm]{Tareas/samplerate.png}	
	}
	\subfloat[Bit resolution]{\includegraphics[height=6cm]{Tareas/bitrate.png}}
\end{figure}






