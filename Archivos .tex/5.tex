\begin{center}
	\LARGE\bfseries Ley A
\end{center}

La característica de compresión recomendada por el UIT-T para Europa y la mayor parte del
resto del mundo se denomina característica  \textbf{Ley A}. Esta característica tiene las mismas
características básicas y ventajas de implementación que la característica p-law.

La relación $\frac{input}{p} -\frac{output}{p}$,o lallamada caracterpistica de comprensión está dada por:


\begin{equation}
	F_A(x)= \left\lbrace \begin{array}{ll}
			sgn(x)\left[ \frac{A|x|}{1+\ln(A)}\right] & 0 \leq |x| < \frac{1}{A}\\
			 	sgn(x)\left[ \frac{1+ \ln(A|x|)}{1+\ln(A)}\right] &  \frac{1}{A} \leq |x| < 1\\
		\end{array}
	\right.
\end{equation}


Y la inversa o característica de expansión por:

\begin{equation}
	F^{-1}_A(y)= \left\lbrace \begin{array}{ll}
		sgn(y)\left[ \frac{|y|+[1+\ln(A)]}{A}\right] & 0 \leq |y| < \frac{1}{1+\ln(A)}\\
		sgn(y)\left[ \frac{1+ \ln(A|x|)}{1+\ln(A)}\right] &  \frac{1}{A} \leq |x| < 1\\
	\end{array}
	\right.
\end{equation}

Cuando $A=1$ no hay compresión.

 La primera porción de la característica de la ley A es lineal por definición. \\
 
 En total, hay ocho
 segmentos positivos y ocho negativos. Los dos primeros segmentos de cada polaridad (cuatro en
 en total) son colineales y, por lo tanto, a veces se consideran como un segmento recto.
 Por lo tanto, la aproximación segmentada de la característica de la ley A se denomina a veces
 como una "aproximación de 13 segmentos". \\
 
 Cuanto más grande es el valor de A, mayor es la compresión. y la característica linear se desplaza a la derecha. \\
 
 La señal de entrada está acotada en $[0,1]$.
 
 \begin{figure}[!htbp]
 	\subfloat{
 	\includegraphics[height=7cm]{Tareas/Alaw.png}	
 	}
 \subfloat{\includegraphics[height=7cm]{Tareas/alawpro}}
 \end{figure}
 
 

