\section{Introducción}

Codificación M-aria
M-ario (eme ario) es un término derivado de la palabra binario. M sólo es un dígito que representa la cantidad de condiciones o combinaciones posibles para determinada cantidad de variables
binarias. Las dos técnicas de modulación digital que se han descrito hasta ahora (FSK binaria y
BPSK) son sistemas binarios; codifican bits individuales y sólo hay dos condiciones posibles de
salida. La FSK produce 1 lógico o frecuencia de marca, o un 0 lógico o frecuencia de espacio,
y la BPSK produce una fase de 1 lógico o una fase de 0 lógico. Los sistemas FSK y BPSK son
M-arios en los que M  2.
Muchas veces conviene, en la modulación digital, codificar a un nivel mayor que el binario (que a veces se dice más allá del binario, o más alto que el binario). Por ejemplo, un sistema PSK (PSK  manipulación por desplazamiento de fase) con cuatro fases de salida posibles
es un sistema M-ario en el que M  4. Si hay ocho fases posibles de salida, M  8, etcétera. La
cantidad de condiciones de salida se calcula con la ecuación

\begin{equation}
	N=\log_2M
\end{equation}
Donde:
N  cantidad de bits codificados
M  cantidad de condiciones posibles de salida con N bit

\section{4-PSK}

Resultados

\begin{figure}
	\centering
	\incl
\end{figure}