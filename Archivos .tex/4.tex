\begin{center}
	\LARGE{\textbf{Tarea 3}} \\
	Gallegos Ruiz Diana Abigail
	
\end{center}

Apartir de las ecuaciones vistas en clase:
\begin{equation}
	IDQ=\frac{2A}{q}
\end{equation}

\begin{equation}
	Q=\frac{1}{12}q^2 \label{dos}
\end{equation}

Se despejó el intervalo de cuantificación a partir de la ec \ref{dos}.

\begin{equation*}
	q^2=\frac{6A^2}{SQR}
\end{equation*}

\begin{equation*}
	q=\sqrt{\frac{6A^2}{SQR}}
\end{equation*}
 
 El código que se utilizó para resolver el ejercicio fue:
 \vspace{1cm}
 
\lstinputlisting[language=Matlab]{Tareas/Tarea3.m}

\newpage

\begin{figure}[!htbp]
	\centering
	\includegraphics[scale=0.5]{Tareas/ultima.png}
	\caption{Señal en el dominio del tiempo}
\end{figure}

De donde

La potencia promedio y el IDQ resultaron:
\begin{figure}[!htbp]
	\centering
	\includegraphics[scale=0.3]{Tareas/promedio.png}
	\caption{Señal en el dominio del tiempo}
\end{figure}

Por lo tanto se requieren de 3 bits para cuantificar la señal sin sobrecargar el IDQ y el valor promedio fue de $P_m=0.0011 W$.


	

