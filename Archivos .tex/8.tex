Las técnicas de sincronización se clasifican según 2 aspectos:

\begin{itemize}
	\item \textbf{Conocimiento previo de los símbolos transmitidos dentro del intervalo de la señal observada:}
	
	\begin{enumerate}
		\item \emph{Asistida por datos data-aided (DA):}
		\\Basado en secuencias de símbolos de referencia conocidas por el
		receptor (señales de entrenamiento, preámbulos/midámbulos, frecuencias
		frecuencias piloto, etc.).
		\item \emph{Dirigido por la decisión Decision-directed (DD)}:¿¿ Utiliza como referencia los valores de los símbolos detectados.
		
		\item \emph{Sin ayuda de datos Non-data-aided (NDA)  (NDA)}:\\ conocidos o detectados. 
	\end{enumerate}

\item \textbf{Conocimiento previo de los símbolos transmitidos dentro del intervalo de la señal observada:}

\begin{enumerate}
	\item  \emph{Lazo abierto feedforward:} \\ Cuanto se hace una estimación de un parámetro de sincronismo y a continuación esta se uda para sincronizaar la señal.
	\item  \emph{Lazo cerrado feedback:}\\ Cuando las muestras se procesan secuencialmente alternando estimaciones y correciones .
\end{enumerate}

\end{itemize}
