% This LaTeX was auto-generated from MATLAB code.
% To make changes, update the MATLAB code and export to LaTeX again.

\documentclass{article}

\usepackage[utf8]{inputenc}
\usepackage[T1]{fontenc}
\usepackage{lmodern}
\usepackage{graphicx}
\usepackage{color}
\usepackage{hyperref}
\usepackage{amsmath}
\usepackage{amsfonts}
\usepackage{epstopdf}
\usepackage[table]{xcolor}
\usepackage{matlab}

\sloppy
\epstopdfsetup{outdir=./}
\graphicspath{ {./8PSKREPOSRTE_images/} }

\matlabhastoc

\begin{document}

\label{T_A178AABD}
\matlabtitle{M-PSK}

\begin{par}
\begin{flushleft}
Gallegos Ruiz Diana Abigail
\end{flushleft}
\end{par}

\matlabtableofcontents{Table of Contents}

\vspace{1em}
\label{H_D302C281}
\matlabheading{Introducción}


\vspace{1em}
\begin{par}
\begin{flushleft}
Codificación M-aria
\end{flushleft}
\end{par}

\begin{par}
\begin{flushleft}
M-ario (eme ario) es un término derivado de la palabra binario. M sólo es un dígito que representa la cantidad de condiciones o combinaciones posibles para determinada cantidad de variables
\end{flushleft}
\end{par}

\begin{par}
\begin{flushleft}
binarias. Las dos técnicas de modulación digital que se han descrito hasta ahora (FSK binaria y
\end{flushleft}
\end{par}

\begin{par}
\begin{flushleft}
BPSK) son sistemas binarios; codifican bits individuales y sólo hay dos condiciones posibles de
\end{flushleft}
\end{par}

\begin{par}
\begin{flushleft}
salida. La FSK produce 1 lógico o frecuencia de marca, o un 0 lógico o frecuencia de espacio,
\end{flushleft}
\end{par}

\begin{par}
\begin{flushleft}
y la BPSK produce una fase de 1 lógico o una fase de 0 lógico. Los sistemas FSK y BPSK son
\end{flushleft}
\end{par}

\begin{par}
\begin{flushleft}
M-arios en los que M  2.
\end{flushleft}
\end{par}

\begin{par}
\begin{flushleft}
Muchas veces conviene, en la modulación digital, codificar a un nivel mayor que el binario (que a veces se dice más allá del binario, o más alto que el binario). Por ejemplo, un sistema PSK (PSK  manipulación por desplazamiento de fase) con cuatro fases de salida posibles
\end{flushleft}
\end{par}

\begin{par}
\begin{flushleft}
es un sistema M-ario en el que M  4. Si hay ocho fases posibles de salida, M  8, etcétera. La
\end{flushleft}
\end{par}

\begin{par}
\begin{flushleft}
cantidad de condiciones de salida se calcula con la ecuación:
\end{flushleft}
\end{par}


\vspace{1em}
\begin{par}
$$N=\log_2 M$$
\end{par}

\begin{par}
\begin{flushleft}
Donde:
\end{flushleft}
\end{par}

\begin{par}
\begin{flushleft}
N  cantidad de bits codificados
\end{flushleft}
\end{par}

\begin{par}
\begin{flushleft}
M  cantidad de condiciones posibles de salida con N bit
\end{flushleft}
\end{par}

\label{H_E337497A}
\matlabheading{4-PSK}

\begin{par}
\begin{flushleft}
Gallegos Ruiz Diana Abigail
\end{flushleft}
\end{par}

\begin{matlabcode}
clc
close all
clear all


% -------------- SECUENCIA DE BITS - ----------
num = randperm(255,15);
numb=cellstr(dec2bin(num));
for i=1:15 %Concatenar secuencia de bits
    if i==1 || i==2
        vec=strcat(numb(1),numb(2));
    else
        vec=strcat(vec,numb(i));
    end    
end

vec=char(vec);

%-------GRÁFICA DE LA SECUENCIA GENERADA  -------------------

n=length(vec)*40;

%Valor de a
a=1/sqrt(2);

p1=sin(2*pi*n);
p2=cos(2*pi*n);
FI=[];
FQ=[];

for i=1:length(vec)
    fdn(i)=str2num(vec(i));
end


for i=1:2:length(vec)
        switch vec(1,i:i+1)
            case '11'
                FI=[FI, ones(1,40)*a];
                FQ=[FQ, ones(1,40)*a];
            case '01'
                FI=[FI ,ones(1,40)*a*-1];
                FQ=[FQ , ones(1,40)*a];
            case '00'
                FI=[FI ,ones(1,40)*a*-1];
                FQ=[FQ,ones(1,40)*a*-1];
            case '10'
                FI=[FI ,ones(1,40)*a];
                FQ=[FQ,ones(1,40)*a*-1];
        end
 end

figure(1)
tiledlayout(5,1)
nexttile
plot(fdn)
ylim([-0.1,1.2])
title('Señal digital f_n(n)')
nexttile
plot(fdn)
xlim([0,8])
title(strcat('Fragmento de secuencia: ',vec(1,1:8)))

nexttile
plot(FI)
xlim([0,200])
ylim([-1.2,1.2])
title('f_I(n)')

nexttile  
plot(FQ)
xlim([0,200])
ylim([-1.2,1.2])
title('f_Q(n)')

FDN=fft(fdn);
nexttile;
plot(abs(FDN));
title('F_n(\omega)')


%------------------ MODULACIÓN Y RUIDO ------------------------------------
n=1:length(vec)*20;
FIm= sin(2*pi*n/13).*FI;
FQm=cos(2*pi*n/13).*FQ;
especI=fft(FIm);
especQ=fft(FQm);

\end{matlabcode}

\begin{par}
\begin{flushleft}
El ruido de $\sigma =0\ldotp 2$, se aprecia en el diagrama de constelación y se observa como tienen a acercarse a los cuadrantes.
\end{flushleft}
\end{par}

\begin{matlabcode}
 Y=(FIm+FQm)+randn(1,2400)*0.2;
 
 especY=fft(Y);

figure(2)
tiledlayout(3,1)
nexttile
plot(FIm)
title('f_I(n) modulada')
nexttile
plot(FIm)
xlim([0,200])
nexttile
plot(FQm)
plot(abs(especI));



figure(3)
tiledlayout(3,1)
nexttile
plot(FQm)
title('f_Q(n) modulada')
nexttile
plot(FQm)
xlim([0,200])
nexttile
plot(abs(especQ))



figure(4)
tiledlayout(3,1)
nexttile
plot(Y)
title('4 PSK')
nexttile
plot(Y)
xlim([0,200])
nexttile
plot(abs(especY))

\end{matlabcode}

\begin{par}
\begin{flushleft}
En la demodulación se pone en práctica la detección de portadoras en fase y cuadratura
\end{flushleft}
\end{par}

\begin{matlabcode}
%------------------------ DEMODULACIÓN ------------------------------------
demFI= Y.*sin(2*pi*n/13);
demFQ=Y.*cos(2*pi*n/13);


%demFI=FIm.*sin(2*pi*n/13);
%demFQ=FQm.*cos(2*pi*n/13);


figure(5)
tiledlayout(2,1)
nexttile
plot(abs(fft(demFI)))
title('F_I(\omega)')
nexttile
plot(abs(fft(demFQ)))
title('F_Q(\omega)')


%----------------------------- FILTRADO -----------------------------------
wc=1/8; % Frecuencia de corte del filtro debería ser menor que la portadora
[N,D] = butter(1,wc,'low');
fIrec=filter(N,D,demFI);
fQrec=filter(N,D,demFQ);


figure(6)
tiledlayout(2,1)
nexttile
plot(fIrec)
title('F_I(t)')
nexttile
plot(fQrec)
title('F_Q(t)')

Frec=fIrec+fQrec;

fidiagrama=detectar(fIrec);
fqdiagrama=detectar(fQrec);

figure(7)
    plot(fidiagrama, fqdiagrama, '.', 'markersize', 20)
title('Diagrama de constelación');

vecrec=[];

prodI = convertir(fidiagrama); 
prodQ = convertir(fqdiagrama);

for i=1:120
    vecrec=[vecrec prodI(i) prodQ(i)];
end

figure(8)
plot(vecrec);


%FUNCIÓN PARA DETECTAR 1 & 0 
function detec = detectar(x)
k=length(x);
detec = zeros(1,k/40);
r=1;
while(r<120)
    for j=0:40:k-40
        avera=0;
        for i=1:40
          avera= x(j+i);
        end
        avera=avera/40;
        detec(r)=avera;
        r=r+1;
    end
end

end

%FUNCIÓN PARA CONVERTIR A BINARIO
function con = convertir(s)
a=1/sqrt(2);
con=[];
    for i=1:length(s)
            if s(i)==a
              con=[con,'1'];
            elseif s(i)==a*-1
              con=[con,'0'];   
            else
              con=[con,'2'];
            end 
    end
end

\end{matlabcode}


\vspace{1em}
\label{H_34CD6AFA}
\matlabheading{8-PSK}


\vspace{1em}
\begin{par}
\begin{flushleft}
Para el 8-PSK  se implementaron los dos convertidores analógicos digitales serie paralelo. Este programa funciona con el uso de funciones lógicas.
\end{flushleft}
\end{par}


\vspace{1em}
\begin{matlabcode}
clc
close all
clear all


% -------------- SECUENCIA DE BITS - ----------
num = randperm(255,15);
numb=cellstr(dec2bin(num));
for i=1:15 %Concatenar secuencia de bits
    if i==1 || i==2
        vec=strcat(numb(1),numb(2));
    else
        vec=strcat(vec,numb(i));
    end    
end

vec=char(vec);

%-------GRÁFICA DE LA SECUENCIA GENERADA  -------------------



%Valor de a & b 
a = 0.541;
b = 1.307;


FI=[];
FQ=[];

for i=1:length(vec)
    fdn(i)=str2num(vec(i));
end

%------------------- CONVERTIDOR ------------------------------------------
inde=1;
for i=1:3:length(fdn)
         I=fdn(1,i);
         Q=fdn(1,i+1);
         C=fdn(1,i+2);
         %PARA C
         FIver(inde)= I;
         Cver (inde)=C;
         switch num2str(([I C]))
            case '0  0'
                FI=[FI, ones(1,40)*a*-1];
            case '0  1'
                FI=[FI ,ones(1,40)*b*-1];
            case '1  0'
                FI=[FI ,ones(1,40)*a];
            case '1  1'
                FI=[FI ,ones(1,40)*b];
        end
          
         %PARA C NEGADA
          C=double(not(C));
%          FQver(inde)= [Q C];
         switch num2str(([Q C]))
            case '0  1'
                 FQ=[FQ, ones(1,40)*b*-1];
            case '0  0'
                 FQ=[FQ ,ones(1,40)*a*-1];
            case '1  1'
                 FQ=[FQ ,ones(1,40)*b];
            case '1  0'
                 FQ=[FQ , ones(1,40)*a];
         end
         inde=inde+1;
end




figure(1)
tiledlayout(2,1)
nexttile
plot(fdn,'Color','#8A0868','LineWidth',1.2)
ylim([-0.1,1.2])
title('Señal digital f_n(t)')
nexttile
plot(abs(fft(fdn)),'Color','#8A0868','LineWidth',1.2)
title('F_n(\omega)')
\end{matlabcode}
\begin{center}
\includegraphics[width=\maxwidth{56.196688409433015em}]{figure_0.png}
\end{center}
\begin{matlabcode}

figure(2)
tiledlayout(2,1)
nexttile
plot(FI,'LineWidth',1.3,'Color','#f59e2c')
ylim([-b-0.2,b+0.2])
title(' Salida F_I(t) del CAD ')
nexttile
plot(FQ,'LineWidth',1.3,'Color','#f59e2c')
ylim([-b-0.2,b+0.2])
title(' Salida F_Q(t) del CAD ')
\end{matlabcode}
\begin{center}
\includegraphics[width=\maxwidth{56.196688409433015em}]{figure_1.png}
\end{center}
\begin{matlabcode}

%-------------------------- MODULACIÓN ------------------------------------
n=1:length(FI);
p1=sin(2*pi*n/13);
p2=cos(2*pi*n/13);


fmI=FI.*p1;
fmQ=FQ.*p2;

%FI
figure(3)
tiledlayout(3,1);
nexttile
plot(fmI,'LineWidth',1.3,'Color','#51bf30')
ylim([-b-0.2,b+0.2])
title('F_{IM}(t)')
nexttile
plot(fmI,'LineWidth',1.3,'Color','#51bf30')
xlim([0,220])
title('F_{IM}(t)')
nexttile
plot(abs(fft(fmI)),'LineWidth',1.3,'Color','#51bf30')
title('F_{IM}(\omega) modulada')
\end{matlabcode}
\begin{center}
\includegraphics[width=\maxwidth{56.196688409433015em}]{figure_2.png}
\end{center}
\begin{matlabcode}


%FQ
figure(4)
tiledlayout(3,1);
nexttile
plot(fmQ,'LineWidth',1.3,'Color','#4f7082')
ylim([-b-0.2,b+0.2])
title('F_{QM}(t)')
nexttile
plot(fmQ,'LineWidth',1.3,'Color','#4f7082')
xlim([0,220])
title('F_{QM}(t)')
nexttile
plot(abs(fft(fmQ)),'LineWidth',1.3,'Color','#4f7082')
title('F_{QM}(\omega) modulada')
\end{matlabcode}
\begin{center}
\includegraphics[width=\maxwidth{56.196688409433015em}]{figure_3.png}
\end{center}
\begin{matlabcode}

%-------------------------------- RUIDO -----------------------------------
s=fmQ+fmI;
r = 0.05*randn(1,length(s));
s = s+r;
figure(5)
tiledlayout(2,1);
nexttile
plot(s,'LineWidth',1.3,'Color','#e65091')
title('8-PSK(t) con \sigma=0.5')
nexttile
plot(abs(fft(s)),'LineWidth',1.3,'Color','#e65091')
title('8-PSK(\omega) con \sigma=0.5')
\end{matlabcode}
\begin{center}
\includegraphics[width=\maxwidth{56.196688409433015em}]{figure_4.png}
\end{center}
\begin{matlabcode}



%-------------------- RECEPTOR--- -----------------------------------------

 % ---  DEMODULACIÓN   ---

demFI= s.*sin(2*pi*n/13);
demFQ=s.*cos(2*pi*n/13);

wc=1/8; % Frecuencia de corte del filtro debería ser menor que la portadora
[N,D] = butter(1,wc,'low');
fIrec=filter(N,D,demFI);
fQrec=filter(N,D,demFQ);

figure(6)
tiledlayout(4,1)
nexttile
plot(fIrec,'LineWidth',1.3,'Color','#bd313f')
title('F_{IM}(t) recuperada')
nexttile
plot(abs(fft(fIrec)),'LineWidth',1.3,'Color','#bd313f')
title('F_{IM}(\omega) recuperada')
nexttile
plot(fQrec,'LineWidth',1.3,'Color','#bd31bb' )
title('F_{QM}(t) recuperada')
nexttile
plot(abs(fft(fIrec)),'LineWidth',1.3,'Color','#bd31bb')
title('F_{QM}(\omega) recuperada')
\end{matlabcode}
\begin{center}
\includegraphics[width=\maxwidth{56.196688409433015em}]{figure_5.png}
\end{center}
\begin{matlabcode}

%------------------------     DETECTOR  -----------------------
 fidiagrama=detectar(fIrec);
fqdiagrama=detectar(fQrec);

z=1
\end{matlabcode}
\begin{matlaboutput}
z = 1
\end{matlaboutput}
\begin{matlabcode}
for i=1:2:80
   
diagramaver2(z)=fidiagrama(i+1);
diagramaver1(z)=fidiagrama(i);
 z=z+1;
end

%[diagramaver1' FIver' Cver' diagramaver2']


%FUNCIÓN PARA DETECTAR 1 & 0 
\end{matlabcode}


\begin{matlabcode}
function detec = detectar(x)
k=length(x);
detec = zeros(1,(k/40));
r=1;
while(r<40)
    for j=0:20:k-20
        avera=0;
        for i=1:20
          avera= x(j+i);
        end
        avera=avera/40;
         if avera <= 0 
             detec(r)= 0;
         else
             detec(r)= 1;
         end    
        r=r+1;
    end
end
end
 
\end{matlabcode}

\label{H_C0AF11DB}
\vspace{1em}

\label{H_1D6338AE}
\matlabheading{Conclusiones}


\vspace{1em}
\begin{par}
\begin{flushleft}
Con el aumento de bits M, mayor es la complejidad de programarlo en MatLab, sin embargo en la práctica, se vuelven sistemas con mayor velocidad de transmisión y con tendecia a errores que pueden ser detectados con el uso de la Teoría de la Información. 
\end{flushleft}
\end{par}

\end{document}
