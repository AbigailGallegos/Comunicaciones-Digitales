\begin{center}
	\textbf{\Huge Tarea 3}
\end{center}


Si la señal de entrada se cuantifica en $n$ niveles, espaciados por un incremento de amplitud a, la distribución razonables de los niveles de cuantificación son:

\begin{figure} [!htbp]
	\centering
	\includegraphics[scale=0.8]{1.png}
\end{figure} 

El error de cuantificación $e_{q}(n)$ está siempre en el rango $-\frac{\Delta}{2}$ a $- \frac{\Delta}{2}$ mientras la señal analógica de entrada se encuentre dentro del rango del cuantificador:

\begin{equation}
	- \frac{\Delta}{2} < e_q(n) < \frac{\Delta}{2}
\end{equation}

el error está limitado en magnitud, para estabilizar la pérdida de la información. Este es el resultado de la ambigüedad introducida por la cuantificación, dado que a todas las muestras a un intervalo inferior de un determinado nivel se les asignan el mismo valor. \\

