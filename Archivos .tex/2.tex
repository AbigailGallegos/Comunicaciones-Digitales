
Para la señal:
\begin{equation}
	f_1(t)= \sin (2\pi \ 10 t)
\end{equation}

Se obtuvieron las siguientes gráficas:

\begin{figure}[!htbp]
	\centering
	\includegraphics[scale=0.4]{img2/11.png}
	\caption{$f_s= 100 Hz$}
\end{figure}

\begin{figure}[!htbp]
	\centering
	\includegraphics[scale=0.4]{img2/12.png}
	\caption{$f_s= 35 Hz$}
\end{figure}

\newpage

\begin{figure}[!htbp]
	\centering
	\includegraphics[scale=0.5]{img2/13.png}
	\caption{$f_s= 30 Hz$}
\end{figure}

\begin{figure}[!htbp]
	\centering
	\includegraphics[scale=0.5]{img2/14.png}
	\caption{$f_s= 25 Hz$}
\end{figure}
 
 \newpage
 
 \begin{figure}[!htbp]
 	\centering
 	\includegraphics[scale=0.5]{img2/15.png}
 	\caption{$f_s= 20 Hz$}
 \end{figure}
 
 \begin{figure}[!htbp]
 	\centering
 	\includegraphics[scale=0.5]{img2/16.png}
 	\caption{$f_s= 15 Hz$}
 \end{figure}

\newpage

\begin{figure}[!htbp]
	\centering
	\includegraphics[scale=0.9]{img2/17.png}
	\caption{$f_s= 10 Hz$}
\end{figure}

\begin{figure}[!htbp]
	\centering
	\includegraphics[scale=0.5]{img2/18.png}
	\caption{$f_s= 8 Hz$}
\end{figure}


\newpage

Para la señal:
\begin{equation}
	f_1(t)= \sin (2\pi \ 30 t)
\end{equation}

Se obtuvieron las siguientes gráficas:

\begin{figure}[!htbp]
	\centering
	\includegraphics[scale=0.4]{img2/21.png}
	\caption{$f_s= 100 Hz$}
\end{figure}

\begin{figure}[!htbp]
	\centering
	\includegraphics[scale=0.4]{img2/22.png}
	\caption{$f_s= 35 Hz$}
\end{figure}

\newpage

\begin{figure}[!htbp]
	\centering
	\includegraphics[scale=0.5]{img2/23.png}
	\caption{$f_s= 30 Hz$}
\end{figure}

\begin{figure}[!htbp]
	\centering
	\includegraphics[scale=0.5]{img2/24.png}
	\caption{$f_s= 25 Hz$}
\end{figure}

\newpage

\begin{figure}[!htbp]
	\centering
	\includegraphics[scale=0.5]{img2/25.png}
	\caption{$f_s= 20 Hz$}
\end{figure}

\begin{figure}[!htbp]
	\centering
	\includegraphics[scale=0.5]{img2/26.png}
	\caption{$f_s= 15 Hz$}
\end{figure}

\newpage

\begin{figure}[!htbp]
	\centering
	\includegraphics[scale=0.5]{img2/27.png}
	\caption{$f_s= 10 Hz$}
\end{figure}

\begin{figure}[!htbp]
	\centering
	\includegraphics[scale=0.5]{img2/28.png}
	\caption{$f_s= 8 Hz$}
\end{figure}

\newpage

Para la señal:
\begin{equation}
	f_1(t)= \sin (2\pi \ 30 t)
\end{equation}

Se obtuvieron las siguientes gráficas:

\begin{figure}[!htbp]
	\centering
	\includegraphics[scale=0.4]{img2/31.png}
	\caption{$f_s= 100 Hz$}
\end{figure}

\begin{figure}[!htbp]
	\centering
	\includegraphics[scale=0.4]{img2/32.png}
	\caption{$f_s= 35 Hz$}
\end{figure}

\newpage

\begin{figure}[!htbp]
	\centering
	\includegraphics[scale=0.5]{img2/33.png}
	\caption{$f_s= 30 Hz$}
\end{figure}

\begin{figure}[!htbp]
	\centering
	\includegraphics[scale=0.5]{img2/34.png}
	\caption{$f_s= 25 Hz$}
\end{figure}

\newpage

\begin{figure}[!htbp]
	\centering
	\includegraphics[scale=0.5]{img2/35.png}
	\caption{$f_s= 20 Hz$}
\end{figure}

\begin{figure}[!htbp]
	\centering
	\includegraphics[scale=0.5]{img2/36.png}
	\caption{$f_s= 15 Hz$}
\end{figure}

\newpage

\begin{figure}[!htbp]
	\centering
	\includegraphics[scale=0.5]{img2/37.png}
	\caption{$f_s= 10 Hz$}
\end{figure}

\begin{figure}[!htbp]
	\centering
	\includegraphics[scale=0.5]{img2/38.png}
	\caption{$f_s= 8 Hz$}
\end{figure}

\newpage

Código:

\begin{lstlisting}[language=Matlab]
	close all
	clear all 
	
	%Tiempo
	ts=[1/100 , 1/35 , 1/30 , 1/25 , 1/20 , 1/15 , 1/10 , 1/8];
	n=800;
	
	for i=1:length(ts)
	t=0.0001:ts(i):2;
	
	
	%SENALES MUESTREADAS 
	
	%f1= sin(2*pi*10*t); %Tiempo continuo
	%f1= sin(2*pi*20*t); 
	f1= sin(2*pi*30*t);
	
	
	% TRANSFORMADA DE FOURIER DE LAS SENALES
	
	F1=0;
	%F2=0;
	%F3=0;
	
	F1=fft(f1,1/ts(i));
	
	invFI=ifft(F1);
	w= linspace(-25*pi,25*pi,1/ts(i));
	
	figure(i)
	tiledlayout('flow')
	% Top plot
	nexttile
	plot(t,f1)
	title(strcat('f(t) con ts=',num2str(ts(i))))
	grid on
	xlabel('t [s]')
	
	% Bottom plot
	nexttile
	plot(w/(2*pi),abs(F1))
	title('Transformada de Fourier')
	grid on
	xlabel('w [Hz]')
	
	t=linspace(0.0001,2,1/ts(i));
	% Top plot
	nexttile
	plot(t,invFI)
	title('Senal recuperada')
	grid on
	xlabel('t [s]')
	end
	
\end{lstlisting}


 


